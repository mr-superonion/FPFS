\documentclass[a4paper,fleqn,usenatbib]{mnras}
\emergencystretch=1.em
\pdfminorversion=5
\newcommand \redColor{}
\newcommand \blueColor{}
\renewcommand\labelenumi{(\roman{enumi})}
\renewcommand\theenumi\labelenumi

\usepackage{graphicx,subfigure}	% Including figure files
\usepackage{epstopdf}
\usepackage{amsmath,amssymb}	% Advanced maths commands

\usepackage{natbib}
\bibliographystyle{mnras}
%\usepackage[breaklinks]{hyperref}
\urlstyle{same}
\usepackage{breakurl}
\hypersetup{breaklinks}


\title{FPFS2: Second generation of Fourier Power Function Shapelets}
\author[]{}

\begin{document}
\label{firstpage}
\pagerange{\pageref{firstpage}--\pageref{lastpage}}
\maketitle

\begin{abstract}
\end{abstract}


\begin{keywords}
cosmology: observations -- gravitational lensing: weak
\end{keywords}

\section{Introduction}
\label{sec:Intro}


\section{Method}
\label{sec:Method}

We first introduce the second generation of FPFS shear estimator in Section
\ref{sec_Method_shapelets} and the revision of selection bias in Section
\ref{sec_Method_select}. Then, we summarize the difference between the updated
FPFS algorithm with its first generation.

\subsection{FPFS2 Shear estimator}
\label{sec_Method_shapelets}

\citet{Z08} first proposed to measure shear from the Fourier power function of
galaxy images. The Fourier power function of a galaxy is defined as
\begin{equation} \label{eq:FourPowDef}
\begin{split}
\tilde{f}_o(\vec{k})&=\int f_o(\vec{x}_o) e^{-i\vec{k} \cdot \vec{x}_o} d^2x_o,\\
\tilde{F}_o(\vec{k})&=|\tilde{f}_o(\vec{k})|^2.
\end{split}
\end{equation}

The Fourier power function defined in equation (\ref{eq:FourPowDef}) is
contaminated by the Fourier power function of noise. Although noise on CCD
images (single exposures) does not correlate across pixels \citep{Z15}, noise
on coadd exposures correlates across pixels as an ad hoc warping kernel is used
to convolve CCD images before re-pixelazing onto common coordinates in the
coadding procedure. In this paper, we assume the noise's Fourier power function
is fully known, which is denoted as $\tilde{F}_n$. The noise's Fourier power
function is subtracted from the galaxy's Fourier power function
\begin{equation}
\tilde{F}_r(\vec{k})=\tilde{F}_o(\vec{k})-\tilde{F}_n
\end{equation}

Subsequently, the PSF's Fourier power function, which is denoted as
$\tilde{G}(\vec{k})$, is deconvolved to revise for the smearing from PSF
\begin{equation}\label{PSF deconvolution_fourier}
\tilde{F}(\vec{k})=\frac{\tilde{F}_r(\vec{k})}{\tilde{G}(\vec{k})}.
\end{equation}

Next, the deconvolved galaxy's Fourier power function is projected onto the
polar Shapelet basis vectors \citep{polar_Shapelets} as
\begin{align}\label{Shapelets_decompose}
M_{nm}=\int \chi_{nm}^{*} \tilde{F}(\rho,\phi) \rho d\rho d\phi.
\end{align}

The polar Shapelet basis vectors are defined as
\begin{align*}
\chi_{nm}(\rho,\phi)&=\frac{(-1)^{(n-|m|)/2}}{\sigma^{|m|+1}}\left\lbrace
    \frac{[(n-|m|)/2]!}{\pi[(n+|m|)/2]!}\right\rbrace^\frac{1}{2}\\
    &\times
    \rho^{|m|}L^{|m|}_{\frac{n-|m|}{2}}\left(\frac{r^2}{\sigma^2}\right)e^{-\rho^2/2\sigma^2}
    e^{-im\phi},
\end{align*}
where $L^{p}_{q}$ are the Laguerre Polynomials, $n$ is the radial number and
$m$ is the spin number, and $\sigma$ determines the scale of Shapelet
functions. We denote the ratio between $\sigma$ and the scale radius of PSF
Fourier power function ($r_{\text{pp}}$) as \citep{FPFS-Li2018}
\begin{equation}
\beta=\frac{\sigma}{r_{\text{pp}}}.
\end{equation}
Note $\beta$ determines the effective scale in Fourier space.
\citet{FPFS-Li2018} proposed to set $\beta=0.85$.

The transformations (\citep{polar_Shapelets}) of many useful shapelet modes under the influence of shear are laid out as follows
\begin{equation}\label{Shapelets_Moments_shear_Transform}
\begin{split}
M_{22c}&=\bar{M}_{22c}-\frac{\sqrt{2}}{2}g_1(\bar{M}_{00}-\bar{M}_{40})\\
&+\sqrt{3}g_1 \bar{M}_{44c}+\sqrt{3} g_2 \bar{M}_{44s},\\
M_{22s}&=\bar{M}_{22s}-\frac{\sqrt{2}}{2}g_2(\bar{M}_{00}-\bar{M}_{40})\\
&-\sqrt{3}g_2 \bar{M}_{44c}+\sqrt{3} g_1 \bar{M}_{44s},\\
M_{00} &=\bar{M}_{00}+\sqrt{2}(g_1\bar{M}_{22c}+g_2\bar{M}_{22s}),\\
M_{20} &=\bar{M}_{20}+\sqrt{6}(g_1\bar{M}_{42c}+g_2\bar{M}_{42s}),\\
M_{40} &=\bar{M}_{40}-\sqrt{2}(g_1\bar{M}_{22c}+g_2\bar{M}_{22s})\\
&+2\sqrt{3}(g_1\bar{M}_{62c}+g_2\bar{M}_{62s}),
\end{split}
\end{equation}
where $\bar{M}_{nm}$ represent the intrinsic shapelet modes and $M_{nm}$
represent the sheared shapelet modes.

We define the dimensionless FPFS ellipticity and the corresponding shear
response with these shapelet modes as
\begin{align}\label{ellipticity_define}
e_1=\frac{M_{22c}}{M_{20}+C},\qquad
e_2=\frac{M_{22s}}{M_{20}+C},\\
R_{1}=\frac{\sqrt{2}}{2}\frac{M_{00}-M_{40}}{M_{20}+C}
    +\sqrt{6}\frac{M_{22c}}{M_{20}+C}\frac{M_{42c}}{M_{20}+C},\\
R_{2}=\frac{\sqrt{2}}{2}\frac{M_{00}-M_{40}}{M_{20}+C}
    +\sqrt{6}\frac{M_{22s}}{M_{20}+C}\frac{M_{42s}}{M_{20}+C},
\end{align}
$M_{nmc}$ and $M_{nms}$ are used to denote the real and imaginary part of
$M_{nm}$ when $m>0$. The constant $C=\nu \sigma(M_{20})$ the weighting
parameter which adjusts weight between galaxies with different luminosity and
reduces noise bias, and $\nu$ is termed weighting ratio \citep{FPFS-Li2018}.

With the definition of average response $R= (R_1+R_2)/2$, the final shear
estimator is
\begin{equation}
\gamma_{1,2} =-\frac{\left\langle e_{1,2}
\right\rangle}{\left\langle R \right\rangle}.
\end{equation}

\subsection{Noise bias}
\label{sec_Method_noise}

The Gaussian noise field is denoted as $n(\vec{x})$ and its Fourier transform
is denoted as $\tilde{n}(\vec{k})$. Accord to the Isserlis' theorem, we have
\begin{equation}\label{eq_IsserlisTheorem}
\begin{split}
    \left\langle\tilde{n}(\vec{k})\tilde{n}(-\vec{k})\right\rangle
    &=\mathcal{P}(\vec{k})\\
    \left\langle\tilde{n}^2(\vec{k})\tilde{n}^2(-\vec{k})\right\rangle
    &=<\tilde{n}^2(\vec{k})><\tilde{n}^2(-\vec{k})>+2\mathcal{P}^2(\vec{k})\\
    &=2\mathcal{P}^2(\vec{k})
\end{split}
\end{equation}

The residual of noise power function is defined as
\begin{equation}
    \tilde{\epsilon}(\vec{k})=\tilde{n}(\vec{k})\tilde{n}(-\vec{k})
    -\mathcal{P}(\vec{k}),
\end{equation}
and the uncertainty of the residual can be calculated by combining with eq.
(\ref{eq_IsserlisTheorem}):
\begin{equation}
    \sigma_{\tilde{\epsilon}}(\vec{k})=\mathcal{P}(\vec{k})
\end{equation}

\subsection{Selection function}
\label{sec_Method_select}

We define the FPFS flux ratio as
\begin{align}\label{select_define}
s = \frac{M_{00}}{M_{20}+C},
\end{align}
and use the FPFS flux ratio as the selection function.
The left panel of Fig. \ref{fig_selectHist} shows the histograms of $s$ with
different setups of $\nu$. In addition, the detection of galaxies is also a
selection process which could cause bias to the shear measurement so we show
the histogram of $s$ ($\nu=4$) for the undetected galaxies on the right panel
of Fig. \ref{fig_selectHist}.

The FPFS flux ratio is also influenced by the shear and the relationship
between the sheared FPFS flux ratio ($s$) and the intrinsic FPFS flux ratio
($\bar{s}$) is
\begin{align}\label{select_transform}
s-\bar{s} &=g_1 (\sqrt{2}\frac{M_{22c}}{M_{20}+C}
    -\sqrt{6}s\frac{M_{42c}}{M_{20}+C}) \\
    &+g_2 (\sqrt{2}\frac{M_{22s}}{M_{20}+C}
    -\sqrt{6}s\frac{M_{42s}}{M_{20}+C}).
\end{align}
$\bar{s}$ is isotropic (spin-0) on the intrinsic plane but $s$ is not.
Therefore, the selection using $s$ as the selection function is not an
isotropic selection on the intrinsic plane. Such selection does not align with
the premise that the intrinsic galaxies have isotropic orientations
statistically and causes selection bias.

\subsection{Algorithm updates}

\section{Image Simulation}
\label{sec:sim}

The HSC-like Bulge+Disk+Knot (BDK) simulation is an HSC version of the BDK
simulation \citep{metacal2}. The simulation is generated by Galsim which is an
open-source image simulation package \citep{GalSim}. We use Sersic models
\citep{Sersic1963} which are fitted to the $25.2$ magnitude limited galaxy
sample from the COSMOS data
\footnote{\url{great3.jb.man.ac.uk/leaderboard/data/public/COSMOS_25.2_training_sample.tar.gz}}
to simulate the bulge and disk of galaxies. The fluxes of these galaxies are
scaled by a factor of $2.587$ to match the fluxes in HSC observation. In order
to avoid repeating the exact parameters, we interpolate the joint radius-flux
distribution by randomly rescaling the radius and flux of the original Sersic
model. To simulate the knots of star formation, we distribute $N$ random points
which statistically obey the Gaussian distribution around the center of the
galaxies, where $N$ is a random number evenly distributed between $50$ and
$100$. The ellipticity of the Gaussian distribution follows the ellipticity of
Sersic model and the half light radius of the Gaussian distribution is fixed to
$2.4$ pixels. The pixel scale of the simulation is set to the HSC pixel scale,
namely $0.168''$. The fraction of the flux of the knots is a random number
evenly distributed between $0\%$ and $10\%$. The galaxies are rotated to random
directions and subsequently sheared by the same shear signal
($g_1=0.02,g2=0.00$). For the HSC-like BDK simulation, we use $g_1$ to
determine the multiplicative bias and use $g_2$ to determine the additive bias.
The galaxy images are convolved with a Moffat PSF \citep{Moffat1969}
\begin{equation}\label{Moffat PSF}
g_{m}(\vec{x})=[1+c(|\vec{x}|/r_p)^2]^{-\beta_m},  \end{equation} where
$c=2^{\frac{1}{\beta_m-1}}-1$ is a constant parameter. The profile of the
Moffat PSF is determined by $\beta_m$, where $\beta_{m}=3.5$. The scale of the
Moffat PSF is determined by its Full Width Half Maximum (FWHM), where
$\rm{FWHM}=0.6$.  The ellipticity of the Moffat PSF is set to
$(e_1=0,e_2=0.025)$. Each convolved galaxy is placed around the center of a
$64\times 64$ stamp. The HSC-like BDK simulation generate $4\times 10^8$
galaxies.

\section{Image boundary}
\label{sec:Boundary}

\subsection{periodic boundary}



\subsection{top-hat}


\subsection{Fourier Lanczos}


\section{Summary and Outlook}
\label{sec:Summary}


\section*{Acknowledgements}

\bibliography{citation}

\appendix

\bsp	% typesetting comment
\label{lastpage}
\end{document}

\end{document}

